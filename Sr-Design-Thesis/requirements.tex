\chapter{Requirements}

The requirements section defines and qualifies what our 3D printer feasibility study and the incubating box must do. We communicated with SE3D to determine the requirements.

\section{Functional Requirements}

Functional requirements define what the system must do. \\
The incubating box system will:
\begin{itemize}
	\item Support timed image capture at a pre-specified image per time interval
	\item Have an interactive user interface
	\item Allow user to monitor and control light, temperature, and camera settings
	\item Allow users to download captured images
	\item Allow users to save settings as custom environments to be used later
\end{itemize}



The 3D printer feasibility study system will:
\begin{itemize}
	\item Show feasibility of auto-homing of SE3D's syringe 
	\item Show feasibility of adding a multi-material printing module to an open source printer
\end{itemize}

% =======================================================================

% =======================================================================

\section{Non-Functional Requirements}
Non-functional requirements define the manner in which the functional requirements need to be achieved. \\
The system will be:
\begin{itemize}
	\item User friendly and intuitive
	\item Safe
	\item Secure
	\item Reliable
\end{itemize} 
% =======================================================================

% =======================================================================

\section{Design Constraints}
Design Constraints are non-functional requirements that constrain the solution instead of the problem. \\
The system must:
\begin{itemize}
	\item Function without being connected to a desktop or laptop computer
	\item Be low-cost
\end{itemize}

