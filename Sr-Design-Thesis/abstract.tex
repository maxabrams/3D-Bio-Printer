\begin{abstract}
The 3D Bioprinter project aims to create a 3D bioprinter that can improve the capabilities of  high school teachers to engage students in STEM education. In order to accomplish this goal, the team is working to expand functionality in SE3D’s product line to allow for a better student and teacher user experience and the execution of more interesting experiments. The 3D Bioprinter project’s main goal is to create a modular incubating box with a variety of sensors to allow for custom environments per experiment, a clear interface to control the settings, and an automatic image capture system.  As the project increases functionality, it also will keep the final deliverable as low cost as possible. These additions to the current SE3D 3D Bioprinter will increase effectiveness in the classroom and allow the target audience, high-school students, to better engage in STEM education activities. \\
\\
Keywords: 3D Printing, Bioprinting, STEM, Education, Control Systems, Incubation 
\end{abstract}
